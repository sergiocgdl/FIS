
%%%%%%%%%%%%%%%%%%%%%%%%%%%%%%%%%%%%%%%%%
% Short Sectioned Assignment LaTeX Template Version 1.0 (5/5/12)
% This template has been downloaded from: http://www.LaTeXTemplates.com
% Original author:  Frits Wenneker (http://www.howtotex.com)
% License: CC BY-NC-SA 3.0 (http://creativecommons.org/licenses/by-nc-sa/3.0/)
%%%%%%%%%%%%%%%%%%%%%%%%%%%%%%%%%%%%%%%%%

%----------------------------------------------------------------------------------------
%	PACKAGES AND OTHER DOCUMENT CONFIGURATIONS
%----------------------------------------------------------------------------------------

\documentclass[paper=a4, fontsize=11pt]{scrartcl} % A4 paper and 11pt font size

% ---- Entrada y salida de texto -----

\usepackage[T1]{fontenc} % Use 8-bit encoding that has 256 glyphs
\usepackage[utf8]{inputenc}
%\usepackage{fourier} % Use the Adobe Utopia font for the document - comment this line to return to the LaTeX default
\usepackage{listings}
\usepackage{xcolor}
\lstset { %
	language=C++,
	frame=tb, % draw a frame at the top and bottom of the code block
	tabsize=4, % tab space width
	showstringspaces=false, % don't mark spaces in strings
	numbers=left, % display line numbers on the left
	commentstyle=\color{green}, % comment color
	keywordstyle=\color{blue}, % keyword color
	stringstyle=\color{red} % string color
}


% ---- Idioma --------

\usepackage[spanish, es-tabla]{babel} % Selecciona el español para palabras introducidas automáticamente, p.ej. "septiembre" en la fecha y especifica que se use la palabra Tabla en vez de Cuadro

% ---- Otros paquetes ----

\usepackage{url} % ,href} %para incluir URLs e hipervínculos dentro del texto (aunque hay que instalar href)
\usepackage{amsmath,amsfonts,amsthm} % Math packages
%\usepackage{graphics,graphicx, floatrow} %para incluir imágenes y notas en las imágenes
\usepackage{graphics,graphicx, float} %para incluir imágenes y colocarlas

% Para hacer tablas comlejas
\usepackage{multirow}
\usepackage{threeparttable}
\usepackage{booktabs}

%\usepackage{sectsty} % Allows customizing section commands
%\allsectionsfont{\centering \normalfont\scshape} % Make all sections centered, the default font and small caps

\usepackage{fancyhdr} % Custom headers and footers
\pagestyle{fancyplain} % Makes all pages in the document conform to the custom headers and footers
\fancyhead{} % No page header - if you want one, create it in the same way as the footers below
\fancyfoot[L]{} % Empty left footer
\fancyfoot[C]{} % Empty center footer
\fancyfoot[R]{\thepage} % Page numbering for right footer
\renewcommand{\headrulewidth}{0pt} % Remove header underlines
\renewcommand{\footrulewidth}{0pt} % Remove footer underlines
\setlength{\headheight}{13.6pt} % Customize the height of the header

\numberwithin{equation}{section} % Number equations within sections (i.e. 1.1, 1.2, 2.1, 2.2 instead of 1, 2, 3, 4)
\numberwithin{figure}{section} % Number figures within sections (i.e. 1.1, 1.2, 2.1, 2.2 instead of 1, 2, 3, 4)
\numberwithin{table}{section} % Number tables within sections (i.e. 1.1, 1.2, 2.1, 2.2 instead of 1, 2, 3, 4)

\setlength\parindent{0pt} % Removes all indentation from paragraphs - comment this line for an assignment with lots of text

\newcommand{\horrule}[1]{\rule{\linewidth}{#1}} % Create horizontal rule command with 1 argument of height


%----------------------------------------------------------------------------------------
%	TÍTULO Y DATOS DEL ALUMNO
%----------------------------------------------------------------------------------------

\title{	
\normalfont \normalsize 
\textsc{\textbf{Fundamentos de Ingeniería del Software (2018-2019)} \\ Doble Grado en Ingeniería Informática Y Matemáticas \\ Universidad de Granada} \\ [25pt] % Your university, school and/or department name(s)
\horrule{0.5pt} \\[0.4cm] % Thin top horizontal rule
\huge Memoria Práctica 1 \\ % The assignment title
\horrule{2pt} \\[0.5cm] % Thick bottom horizontal rule
}

\author{Javier Alcántara García, César Muñoz Reinoso, Sergio Cabezas González de Lara} % Nombre y apellidos

\date{\normalsize\today} % Incluye la fecha actual


%----------------------------------------------------------------------------------------
% DOCUMENTO
%----------------------------------------------------------------------------------------

\begin{document}

\maketitle % Muestra el Título

\newpage %inserta un salto de página

\tableofcontents % para generar el índice de contenidos

\listoffigures

\listoftables

\newpage

%----------------------------------------------------------------------------------------
%	Cuestión 1
%----------------------------------------------------------------------------------------

\section{Objetivos}
\begin{enumerate}
	\item El sistema deberá almacenar y gestionar la información correspondiente a los paquetes durante todo el proceso de envío y recepción.
	\item El sistema automatizará todas las actividades relacionadas con el cálculo del coste de envío y tiempo estimado de llegada.
	3. El seguimiento del pedido se podrá realizar a través de la aplicación.
	4. El sistema podrá gestionar la disponibilidad de transportistas y vehículos destinados a cada envío a nivel nacional.
	5. El sistema permitirá la localización de la oficina más cercana al usuario.
	6. El sistema permitirá recoger valoraciones/quejas de los usuarios con la intención de atenderlas.
	
\end{enumerate} 

\newpage
%----------------------------------------------------------------------------------------
%	Cuestión 2
%----------------------------------------------------------------------------------------

\section{Descripción general del sistema}

\newpage
%----------------------------------------------------------------------------------------
%	Cuestión 3
%----------------------------------------------------------------------------------------

\section{Requisitos}
\subsection{Funcionales}
\subsubsection{Gestión del pedido}
que hago...

\subsubsection{Gestión de cliente}
Creación/modificación de usuario desde el que gestionar todos los pedidos.

I. PERMITIR ENVIO DE PEDIDOS
1. Datos de recepción, entrega, elección de seguro y método de pago. Especificando envío/recogida a domicilio o en oficina.
2. En función de los datos introducidos: opciones y coste de envío.
*DESGLOSAR*

II. MODIFICAR DATOS DEL ENVÍO (ANTES DE SALIDA)

III. PERMITIR SEGUIMIENTO DEL ENVÍO

IV. VALORACIÓN DEL SERVICIO DADO






--------------------
-idea- permite gestionar de forma óptima la furgoneta en la que se envia un paquete atendiendo al destino de ambos.
-idea- el sistema permite recoger las valoraciones de los usuarios para mejorar la gestión.
\newpage
\subsection{No funcionales}

\newpage
\subsection{De información}

\newpage
%----------------------------------------------------------------------------------------
%	Cuestión 4
%----------------------------------------------------------------------------------------

\section{Glosario de términos}

\newpage
%------------------------------------------------

\end{document}
