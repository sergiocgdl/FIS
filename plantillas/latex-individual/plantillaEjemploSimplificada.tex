
%%%%%%%%%%%%%%%%%%%%%%%%%%%%%%%%%%%%%%%%%
% Short Sectioned Assignment LaTeX Template Version 1.0 (5/5/12)
% This template has been downloaded from: http://www.LaTeXTemplates.com
% Original author:  Frits Wenneker (http://www.howtotex.com)
% License: CC BY-NC-SA 3.0 (http://creativecommons.org/licenses/by-nc-sa/3.0/)
%%%%%%%%%%%%%%%%%%%%%%%%%%%%%%%%%%%%%%%%%

%----------------------------------------------------------------------------------------
%	PACKAGES AND OTHER DOCUMENT CONFIGURATIONS
%----------------------------------------------------------------------------------------

\documentclass[paper=a4, fontsize=11pt]{scrartcl} % A4 paper and 11pt font size

% ---- Entrada y salida de texto -----

\usepackage[T1]{fontenc} % Use 8-bit encoding that has 256 glyphs
\usepackage[utf8]{inputenc}
%\usepackage{fourier} % Use the Adobe Utopia font for the document - comment this line to return to the LaTeX default
\usepackage{listings}
\usepackage{xcolor}
\lstset { %
	language=C++,
	frame=tb, % draw a frame at the top and bottom of the code block
	tabsize=4, % tab space width
	showstringspaces=false, % don't mark spaces in strings
	numbers=left, % display line numbers on the left
	commentstyle=\color{green}, % comment color
	keywordstyle=\color{blue}, % keyword color
	stringstyle=\color{red} % string color
}


% ---- Idioma --------

\usepackage[spanish, es-tabla]{babel} % Selecciona el español para palabras introducidas automáticamente, p.ej. "septiembre" en la fecha y especifica que se use la palabra Tabla en vez de Cuadro

% ---- Otros paquetes ----

\usepackage{url} % ,href} %para incluir URLs e hipervínculos dentro del texto (aunque hay que instalar href)
\usepackage{amsmath,amsfonts,amsthm} % Math packages
%\usepackage{graphics,graphicx, floatrow} %para incluir imágenes y notas en las imágenes
\usepackage{graphics,graphicx, float} %para incluir imágenes y colocarlas

% Para hacer tablas comlejas
\usepackage{multirow}
\usepackage{threeparttable}
\usepackage{booktabs}

%\usepackage{sectsty} % Allows customizing section commands
%\allsectionsfont{\centering \normalfont\scshape} % Make all sections centered, the default font and small caps

\usepackage{fancyhdr} % Custom headers and footers
\pagestyle{fancyplain} % Makes all pages in the document conform to the custom headers and footers
\fancyhead{} % No page header - if you want one, create it in the same way as the footers below
\fancyfoot[L]{} % Empty left footer
\fancyfoot[C]{} % Empty center footer
\fancyfoot[R]{\thepage} % Page numbering for right footer
\renewcommand{\headrulewidth}{0pt} % Remove header underlines
\renewcommand{\footrulewidth}{0pt} % Remove footer underlines
\setlength{\headheight}{13.6pt} % Customize the height of the header

\numberwithin{equation}{section} % Number equations within sections (i.e. 1.1, 1.2, 2.1, 2.2 instead of 1, 2, 3, 4)
\numberwithin{figure}{section} % Number figures within sections (i.e. 1.1, 1.2, 2.1, 2.2 instead of 1, 2, 3, 4)
\numberwithin{table}{section} % Number tables within sections (i.e. 1.1, 1.2, 2.1, 2.2 instead of 1, 2, 3, 4)

\setlength\parindent{0pt} % Removes all indentation from paragraphs - comment this line for an assignment with lots of text

\newcommand{\horrule}[1]{\rule{\linewidth}{#1}} % Create horizontal rule command with 1 argument of height


%----------------------------------------------------------------------------------------
%	TÍTULO Y DATOS DEL ALUMNO
%----------------------------------------------------------------------------------------

\title{	
\normalfont \normalsize 
\textsc{\textbf{Fundamentos de Ingeniería del Software (2018-2019)} \\ Doble Grado en Ingeniería Informática Y Matemáticas \\ Universidad de Granada} \\ [25pt] % Your university, school and/or department name(s)
\horrule{0.5pt} \\[0.4cm] % Thin top horizontal rule
\huge \textbf{Práctica 2} \\ Modelo de casos de uso \\ % The assignment title
\horrule{2pt} \\[0.5cm] % Thick bottom horizontal rule
}

\author{Javier Alcántara García\\ César Muñoz Reinoso \\ Sergio Cabezas González de Lara \\ Víctor García Carrera} % Nombre y apellidos

\date{\normalsize\today} % Incluye la fecha actual


%----------------------------------------------------------------------------------------
% DOCUMENTO
%----------------------------------------------------------------------------------------

\begin{document}

\maketitle % Muestra el Título

\newpage %inserta un salto de página

\tableofcontents % para generar el índice de contenidos

\newpage

\section{Introducción}
\section{Jerarquía de casos de uso}
\subsection{Gestión del envío}	
Descripción:\\
Escenarios asociados con el envío de los paquetes.\\
Casos de uso:\\
 Identificar socio, Solicitar envío, Obtener coste, Ver datos del envío, Modificar un envío, Cancelar un envío, Ver fase de envío, Seguir los pedidos, Valorar el envío.\\
Actores:\\
Socio, Cliente, Empleado de oficina.\\
\subsection{Gestión del socios}	
Descripción:\\
Escenarios asociados con la gestión de los socios.\\
Casos de uso:\\
 Dar de alta, Dar de baja, Visualizar datos, Visualizar pedidos, Modificar datos, Identificar socio, Notificar descuentos, Valorar el envío, Ver valoraciones.\\
Actores:\\
Socio, Cliente, Empleado de oficina.\\
\subsection{Gestión del transporte}	
Descripción:\\
Escenarios asociados con la gestión del transporte de los paquetes.\\
Casos de uso:\\
Visualizar furgonetas disponibles, Modificar furgonetas disponibles, Obtener ruta de paquete, Modificar ruta de paquete, Visualizar paquetes en almacén, Recibir paquete en almacén, Sacar paquete de almacén, Distribuir paquetes en furgonetas, Visualizar itinerario de transportista, Modificar itinerario de transportista.\\
Actores:\\
Empleado de oficina, Empleado de almacén, Transportista.\\

\newpage

\section{Glosario de términos}
	\begin{enumerate}
		\item \textit{Ruta de envío}: Se trata de las oficinas y almacenes por los que pasa un paquete hasta llegar al destino.
		\item \textit{Identificador de paquete}: Código que se le asigna a un paquete cuando entra al sistema de envíos de la empresa, permitiendo diferenciarlo de todos los demás y simplificar operaciones como la localización.
		\item \textit{Coste de envío}: Coste variable del servicio de envío en función del peso del paquete, dimensiones, distancia, modo de envío (certificado, urgente, etc) y condición de socio.
		\item \textit{Filtro}: Herramienta externa (ej. Google Maps) que se implementa  en el sistema para comprobar si hay algún error de entrada por parte del usuario que solicita una recogida/entrega de paquete.
		\item \textit{Canal de comunicación}: Medio por el que se transmite la información sobre el envío siendo destinatario el usuario(via e-mail o SMS).
		\item \textit{Valoraciones}: Calificación de 0 a 5 y comentarios respecto a el servicio prestado por parte de la empresa.
		
 	\end{enumerate}
\newpage
%------------------------------------------------

\end{document}
