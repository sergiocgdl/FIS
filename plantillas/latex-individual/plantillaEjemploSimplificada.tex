
\input{preambuloSimple.tex}

\graphicspath{{images/}}

%----------------------------------------------------------------------------------------
%	TÍTULO Y DATOS DEL ALUMNO
%----------------------------------------------------------------------------------------

\title{	
\normalfont \normalsize 
\textsc{\textbf{Fundamentos de Ingeniería del Software (2018-2019)} \\ Doble Grado en Ingeniería Informática Y Matemáticas \\ Universidad de Granada} \\ [25pt] % Your university, school and/or department name(s)
\horrule{0.5pt} \\[0.4cm] % Thin top horizontal rule
\huge \textbf{Práctica 4} \\ Diseño \\ % The assignment title
\horrule{2pt} \\[0.5cm] % Thick bottom horizontal rule
}

\author{Javier Alcántara García\\ César Muñoz Reinoso \\ Sergio Cabezas González de Lara \\ Víctor García Carrera} % Nombre y apellidos

\date{\normalsize\today} % Incluye la fecha actual


%----------------------------------------------------------------------------------------
% DOCUMENTO
%----------------------------------------------------------------------------------------

\begin{document}

\maketitle % Muestra el Título

\newpage %inserta un salto de página

\tableofcontents % para generar el índice de contenidos

\newpage

\section{Modelo de interacción de objetos}
\subsection{Gestión de transporte}
\begin{figure}[H]
	\centering
	\includegraphics[width=16cm]{1}
	\caption{Víctor García Carrera}
\end{figure}
\begin{figure}[H]
	\centering
	\includegraphics[width=16cm]{2}
	\caption{Víctor García Carrera}
\end{figure}
\begin{figure}[H]
	\centering
	\includegraphics[width=16cm]{3}
	\caption{Víctor García Carrera}
\end{figure}
\begin{figure}[H]
	\centering
	\includegraphics[width=16cm]{4}
	\caption{Víctor García Carrera}
\end{figure}
\begin{figure}[H]
	\centering
	\includegraphics[width=16cm]{5}
	\caption{Sergio Cabezas González de Lara}
\end{figure}
\begin{figure}[H]
	\centering
	\includegraphics[width=16cm]{6}
	\caption{Sergio Cabezas González de Lara}
\end{figure}
\begin{figure}[H]
	\centering
	\includegraphics[width=16cm]{7}
	\caption{Sergio Cabezas González de Lara}
\end{figure}
\begin{figure}[H]
	\centering
	\includegraphics[width=16cm]{8}
	\caption{Sergio Cabezas González de Lara}
\end{figure}
\begin{figure}[H]
	\centering
	\includegraphics[width=16cm]{9}
	\caption{César Muñoz Reinoso}
\end{figure}
\begin{figure}[H]
	\centering
	\includegraphics[width=16cm]{10}
	\caption{César Muñoz Reinoso}
\end{figure}
\begin{figure}[H]
	\centering
	\includegraphics[width=16cm]{11}
	\caption{César Muñoz Reinoso}
\end{figure}
\begin{figure}[H]
	\centering
	\includegraphics[width=16cm]{12}
	\caption{César Muñoz Reinoso}
\end{figure}
\newpage
\subsection{Atención al cliente}
\begin{figure}[H]
	\centering
	\includegraphics[width=10cm]{13}
	\caption{Javier Alcántara García}
\end{figure}
\begin{figure}[H]
	\centering
	\includegraphics[width=16cm]{14}
	\caption{Javier Alcántara García}
\end{figure}
\begin{figure}[H]
	\centering
	\includegraphics[width=9cm]{15}
	\caption{Javier Alcántara García}
\end{figure}
\begin{figure}[H]
	\centering
	\includegraphics[width=16cm]{16}
	\caption{Javier Alcántara García}
\end{figure}
\newpage

\section{Diagrama de clases del diseño}
\begin{figure}[H]
	\centering
	\includegraphics[width=16cm]{clases}
\end{figure}

\section{Glosario de términos}
	\begin{enumerate}
		\item \textit{Ruta de envío}: Se trata de las oficinas y almacenes por los que pasa un paquete hasta llegar al destino.
		\item \textit{Identificador de paquete}: Código que se le asigna a un paquete cuando entra al sistema de envíos de la empresa, permitiendo diferenciarlo de todos los demás y simplificar operaciones como la localización.
		\item \textit{Coste de envío}: Coste variable del servicio de envío en función del peso del paquete, dimensiones, distancia, modo de envío (certificado, urgente, etc) y condición de socio.
		\item \textit{Fase de envío}: El transporte de un paquete se divide en diversas fases: \newline - Recepción del paquete (en oficina)
		\newline - Aceptación del envío (periodo antes de que el paquete salga para su transporte para modificar el envío)
		\newline - Transporte del paquete (entre oficinas y almacenes hasta la oficina destino)
		\newline - Llegada a oficina destino
		\newline - Llegada a destino (llegada del paquete a la dirección destino)
		\item \textit{Ruta de transporte libre}: Una ruta de transporte puede estar libre o tener una furgoneta asignada.
		\item \textit{Estado de furgoneta}: Una furgoneta puede encontrarse disponible o realizando un desplazamiento, en cuyo caso incluye la ruta de transporte que está llevando a cabo.
		\item \textit{Estado de paquete}: El estado de un paquete está formado por la fase de envío en la que se encuentra, su situación y ubicación actuales y su ruta de transporte asignada.
		\item \textit{Filtro}: Herramienta externa (ej. Google Maps) que se implementa  en el sistema para comprobar si hay algún error de entrada por parte del usuario que solicita una recogida/entrega de paquete.
		\item \textit{Canal de comunicación}: Medio por el que se transmite la información sobre el envío siendo destinatario el usuario(vía e-mail o SMS).
		\item \textit{Valoraciones}: Calificación de 0 a 5 y comentarios respecto a el servicio prestado por parte de la empresa.
		\item \textit{Almacenista}: Empleado encargado de la gestión de paquetes dentro del almacén.
		\item \textit{Empleado}: Trabajador de la empresa de transportes.
		\item \textit{Oficinista}: Empleado encargado de los tramites administrativos dentro de la empresa así como la gestión general de paquetes.
		\item \textit{Almacenista}: Empleado encargado de la distribución de paquetes en los almacenes.
		\item \textit{Conductor}: Empleado encargado del transporte de paquetes entre almacenes.	
		\item \textit{Repartidor}: Empleado encargado de la recogida y entrega de paquetes entre clientes y almacenes.
		\item \textit{Sucursal}: Oficina o almacén de la empresa.
		\item \textit{Almacén}: Edificio donde se guardan y almacenan los paquetes durante su envío hasta el destino.
		\item \textit{Área personal}: Sitio web al que se accede un cliente registrado tras identificarse y con el que puede acceder a los servicios que ofrece la empresa.
 	\end{enumerate}
\newpage
%------------------------------------------------

\end{document}
