
\input{preambuloSimple.tex}

%----------------------------------------------------------------------------------------
%	TÍTULO Y DATOS DEL ALUMNO
%----------------------------------------------------------------------------------------

\title{	
\normalfont \normalsize 
\textsc{\textbf{Fundamentos de Ingeniería del Software (2018-2019)} \\ Doble Grado en Ingeniería Informática Y Matemáticas \\ Universidad de Granada} \\ [25pt] % Your university, school and/or department name(s)
\horrule{0.5pt} \\[0.4cm] % Thin top horizontal rule
\huge \textbf{Práctica 1} \\ Lista de requisitos \\ % The assignment title
\horrule{2pt} \\[0.5cm] % Thick bottom horizontal rule
}

\author{Javier Alcántara García\\ César Muñoz Reinoso \\ Sergio Cabezas González de Lara \\ Víctor García Carrera} % Nombre y apellidos

\date{\normalsize\today} % Incluye la fecha actual


%----------------------------------------------------------------------------------------
% DOCUMENTO
%----------------------------------------------------------------------------------------

\begin{document}

\maketitle % Muestra el Título

\newpage %inserta un salto de página

\tableofcontents % para generar el índice de contenidos

\listoffigures

\listoftables

\newpage

%----------------------------------------------------------------------------------------
%	Cuestión 1
%----------------------------------------------------------------------------------------

\section{Descripción general del sistema \& Objetivos principales}
El sistema software a desarrollar está destinado a la gestión de una empresa de transportes (Envíamelo S.A). Este sistema debe almacenar toda información relativa al envío de paquetes, su transporte y almacenaje. Junto con aquellos datos de envío por parte del cliente, debe gestionar el estado de los almacenes, con información relativa a los paquetes que dispone y sus rutas de envío, además de la flota de furgonetas y transportistas de los que dispone la empresa para el transporte y reparto, optimizando los viajes para minimizar los tiempos de envío y desplazamientos entre almacenes.
En síntesis, los objetivos principales son los siguientes:

\begin{enumerate}
	\item El sistema deberá almacenar y gestionar la información correspondiente a los paquetes durante todo el proceso de envío y recepción, tal como ruta de envío o estado y situación del paquete.
	\item El sistema automatizará todas las actividades relacionadas con el cálculo del coste de envío y tiempo estimado de llegada. Optimiza la ruta de envío en función de la distancia entre los almacenes origen-destino, la disponibilidad de furgonetas y transportistas y el estado de los almacenes.
	\item El cliente puede seleccionar entre diversos modos de envío y elegir las fechas de recogida y entrega del paquete (una vez en almacen)
	\item EL sistema gestiona el estado de los almacenes y sus paquetes.
	\item El seguimiento del pedido se podrá realizar a través de la aplicación.
	\item El sistema podrá gestionar la disponibilidad de transportistas y vehículos destinados a cada envío a nivel nacional. Gestiona el reparto de paquetes entre furgonetas para agrupar aquellos con destinos cercanos, además de gestionar los conductores de las furgonetas, sus rutas y horarios, optimizando los repartos y desplazamientos.
	\item El sistema permitirá la localización de la oficina más cercana al usuario.
	\item El sistema permitirá recoger valoraciones/quejas de los usuarios con la intención de atenderlas.
	 
\end{enumerate}
\newpage
%----------------------------------------------------------------------------------------
%	Cuestión 3
%----------------------------------------------------------------------------------------

\section{Requisitos}
\subsection{Funcionales}
En esta sección se recogerán las características de alto nivel del sistema que permitirán facilitar las necesidades del usuario. Para facilitar su lectura y análisis, los requisitos se presentan en forma de lista estructurada. 
\subsubsection{Gestión del pedido}
Debemos almacenar información sobre el paquete a transportar y gestionar las devoluciones por parte de los clientes.  
\begin{enumerate}
	\item Localización de este
	\item 
\end{enumerate}

\subsubsection{Gestión de cliente}
Creación/modificación de usuario desde el que gestionar todos los pedidos.

\begin{enumerate}
	\item Permitir envío de pedidos
	\subitem Datos de recepción, entrega, elección de seguro y método de pago. Especificando envío/recogida a domicilio o en oficina.
	\subitem En función de los datos introducidos: opciones y coste de envío.
	\item Modificar datos del envío (antes de salida)
	\item Permitir seguimiento del envío
	\item Valoración del seguimiento dado
\end{enumerate}



--------------------
-idea- permite gestionar de forma óptima la furgoneta en la que se envia un paquete atendiendo al destino de ambos.
-idea- el sistema permite recoger las valoraciones de los usuarios para mejorar la gestión.
\newpage
\subsection{No funcionales}
\subsubsection{Usabilidad}
\begin{enumerate}
	\item Se deberá dar \textit{ayuda} en línea con instrucciones paso a paso para guiar al repartidor en las tareas que debe realizar.
	\item Documentación: resguardo del servicio contratado.
	
\end{enumerate}
\subsubsection{Fiabilidad}

\subsubsection{Rendimiento}
\begin{enumerate}
	\item Para optimizar la recepción y entrega de los paquetes, habrá una flota de furgonetas especializada en la conexión entre almacenes. Cada almacén tendrá a su vez una pequeña flota de furgonetas para la recogida/entrega de paquetes al cliente.
	\item Con la intención de reducir costes de transporte, se implementara un sistema de distribución del reparto de paquetes entre furgonetas. De esta forma se evitarán viajes innecesarios, pudiendo delegar el trabajo al transportista más apropiado en cada caso.
	\item Plazos de entrega: Dependiendo del tipo de envío, los plazos de entrega serán los siguientes: estándar (hasta 3 días hábiles), urgente (hasta 2 días hábiles) y express (hasta 24 horas).
\end{enumerate}

\subsubsection{Soporte}


\subsubsection{Restricciones de implementación}

\subsubsection{Requisitos físicos}
- disponibilidad furgonetas, almacenes y transportistas

--------------------------------------------- \\
extras usabilidad \\
- Actualización por mail/sms del pedido en todo momento. \\
- Aviso al repartidor "furgoneta lista para salir"
fiabilidad \\
extras -- \\
- frente a una perdida/retraso del pedido, se informará al cliente por email/sms y se dará una/s solucion/es a ello.\\
- cómo me recupero ante una perdida de paquete. trackear ultima localizacion de este,etc. \\


\newpage

\subsection{De información}

\newpage
%----------------------------------------------------------------------------------------
%	Cuestión 4
%----------------------------------------------------------------------------------------

\section{Glosario de términos}

\newpage
%------------------------------------------------

\end{document}
