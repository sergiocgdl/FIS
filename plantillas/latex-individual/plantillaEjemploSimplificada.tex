
\input{preambuloSimple.tex}

%----------------------------------------------------------------------------------------
%	TÍTULO Y DATOS DEL ALUMNO
%----------------------------------------------------------------------------------------

\title{	
\normalfont \normalsize 
\textsc{\textbf{Algorítmica I (2018-2019)} \\ Doble Grado en Ingeniería Informática Y Matemáticas \\ Universidad de Granada} \\ [25pt] % Your university, school and/or department name(s)
\horrule{0.5pt} \\[0.4cm] % Thin top horizontal rule
\huge Memoria Práctica 1 \\ % The assignment title
\horrule{2pt} \\[0.5cm] % Thick bottom horizontal rule
}

\author{Javier Alcántara García} % Nombre y apellidos

\date{\normalsize\today} % Incluye la fecha actual


%----------------------------------------------------------------------------------------
% DOCUMENTO
%----------------------------------------------------------------------------------------

\begin{document}

\maketitle % Muestra el Título

\newpage %inserta un salto de página

\tableofcontents % para generar el índice de contenidos

\listoffigures

\listoftables

\newpage

%----------------------------------------------------------------------------------------
%	Cuestión 1
%----------------------------------------------------------------------------------------

\section{Objetivos}
1. El sistema deberá almacenar y gestionar la información correspondiente a los paquetes durante todo el proceso de envío y recepción.
2. El sistema automatizará todas las actividades relacionadas con el cálculo del coste de envío y tiempo estimado de llegada.
3. El seguimiento del pedido se podrá realizar a través de la aplicación.
4. El sistema podrá gestionar la disponibilidad de transportistas y vehículos destinados a cada envío a nivel nacional.
5. El sistema permitirá la localización de la oficina más cercana al usuario.
6. El sistema permitirá recoger valoraciones/quejas de los usuarios con la intención de atenderlas.
 

\newpage
%----------------------------------------------------------------------------------------
%	Cuestión 2
%----------------------------------------------------------------------------------------

\section{Descripción general del sistema}

\newpage
%----------------------------------------------------------------------------------------
%	Cuestión 3
%----------------------------------------------------------------------------------------

\section{Requisitos}
\subsection{Funcionales}
\subsubsection{Gestión del pedido}
que hago...

\subsubsection{Gestión de cliente}
Creación/modificación de usuario desde el que gestionar todos los pedidos.

I. PERMITIR ENVIO DE PEDIDOS
1. Datos de recepción, entrega, elección de seguro y método de pago. Especificando envío/recogida a domicilio o en oficina.
2. En función de los datos introducidos: opciones y coste de envío.
*DESGLOSAR*

II. MODIFICAR DATOS DEL ENVÍO (ANTES DE SALIDA)

III. PERMITIR SEGUIMIENTO DEL ENVÍO

IV. VALORACIÓN DEL SERVICIO DADO






--------------------
-idea- permite gestionar de forma óptima la furgoneta en la que se envia un paquete atendiendo al destino de ambos.
-idea- el sistema permite recoger las valoraciones de los usuarios para mejorar la gestión.
\newpage
\subsection{No funcionales}

\newpage
\subsection{De información}

\newpage
%----------------------------------------------------------------------------------------
%	Cuestión 4
%----------------------------------------------------------------------------------------

\section{Glosario de términos}

\newpage
%------------------------------------------------

\end{document}
