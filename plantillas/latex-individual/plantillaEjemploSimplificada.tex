
\input{preambuloSimple.tex}

\graphicspath{{images/}}

%----------------------------------------------------------------------------------------
%	TÍTULO Y DATOS DEL ALUMNO
%----------------------------------------------------------------------------------------

\title{	
\normalfont \normalsize 
\textsc{\textbf{Fundamentos de Ingeniería del Software (2018-2019)} \\ Doble Grado en Ingeniería Informática Y Matemáticas \\ Universidad de Granada} \\ [25pt] % Your university, school and/or department name(s)
\horrule{0.5pt} \\[0.4cm] % Thin top horizontal rule
\huge \textbf{Práctica 2} \\ Modelo de casos de uso \\ % The assignment title
\horrule{2pt} \\[0.5cm] % Thick bottom horizontal rule
}

\author{Javier Alcántara García\\ César Muñoz Reinoso \\ Sergio Cabezas González de Lara \\ Víctor García Carrera} % Nombre y apellidos

\date{\normalsize\today} % Incluye la fecha actual


%----------------------------------------------------------------------------------------
% DOCUMENTO
%----------------------------------------------------------------------------------------

\begin{document}

\maketitle % Muestra el Título

\newpage %inserta un salto de página

\tableofcontents % para generar el índice de contenidos

\newpage

\section{Descripción extendida de casos de uso}
\begin{figure}[H]
	\centering
	\includegraphics[width=16cm]{1}
\end{figure}
\begin{figure}[H]
	\centering
	\includegraphics[width=16cm]{2}
\end{figure}
\begin{figure}[H]
	\centering
	\includegraphics[width=16cm]{3}
\end{figure}
\begin{figure}[H]
	\centering
	\includegraphics[width=16cm]{4}
\end{figure}
\begin{figure}[H]
	\centering
	\includegraphics[width=16cm]{5}
\end{figure}
\begin{figure}[H]
	\centering
	\includegraphics[width=16cm]{6}
\end{figure}
\begin{figure}[H]
	\centering
	\includegraphics[width=16cm]{7}
\end{figure}
\begin{figure}[H]
	\centering
	\includegraphics[width=16cm]{8}
\end{figure}
\begin{figure}[H]
	\centering
	\includegraphics[width=16cm]{9}
\end{figure}
\begin{figure}[H]
	\centering
	\includegraphics[width=16cm]{10}
\end{figure}
\begin{figure}[H]
	\centering
	\includegraphics[width=16cm]{11}
\end{figure}
\begin{figure}[H]
	\centering
	\includegraphics[width=16cm]{12}
\end{figure}
\begin{figure}[H]
	\centering
	\includegraphics[width=16cm]{13}
\end{figure}
\begin{figure}[H]
	\centering
	\includegraphics[width=16cm]{14}
\end{figure}
\begin{figure}[H]
	\centering
	\includegraphics[width=16cm]{15}
\end{figure}
\begin{figure}[H]
	\centering
	\includegraphics[width=16cm]{16}
\end{figure}
\begin{figure}[H]
	\centering
	\includegraphics[width=16cm]{17}
\end{figure}
\begin{figure}[H]
	\centering
	\includegraphics[width=16cm]{18}
\end{figure}
\begin{figure}[H]
	\centering
	\includegraphics[width=16cm]{19}
\end{figure}
\begin{figure}[H]
	\centering
	\includegraphics[width=16cm]{20}
\end{figure}
\begin{figure}[H]
	\centering
	\includegraphics[width=16cm]{21}
\end{figure}
\begin{figure}[H]
	\centering
	\includegraphics[width=16cm]{22}
\end{figure}
\begin{figure}[H]
	\centering
	\includegraphics[width=16cm]{23}
\end{figure}
\begin{figure}[H]
	\centering
	\includegraphics[width=16cm]{24}
\end{figure}
\begin{figure}[H]
	\centering
	\includegraphics[width=16cm]{25}
\end{figure}
\begin{figure}[H]
	\centering
	\includegraphics[width=16cm]{26}
\end{figure}
\begin{figure}[H]
	\centering
	\includegraphics[width=16cm]{27}
\end{figure}
\begin{figure}[H]
	\centering
	\includegraphics[width=16cm]{28}
\end{figure}
\begin{figure}[H]
	\centering
	\includegraphics[width=16cm]{29}
\end{figure}
\begin{figure}[H]
	\centering
	\includegraphics[width=16cm]{30}
\end{figure}
\begin{figure}[H]
	\centering
	\includegraphics[width=16cm]{31}
\end{figure}
\begin{figure}[H]
	\centering
	\includegraphics[width=16cm]{32}
\end{figure}
\begin{figure}[H]
	\centering
	\includegraphics[width=16cm]{33}
\end{figure}
\begin{figure}[H]
	\centering
	\includegraphics[width=16cm]{34}
\end{figure}
\begin{figure}[H]
	\centering
	\includegraphics[width=16cm]{35}
\end{figure}
\begin{figure}[H]
	\centering
	\includegraphics[width=16cm]{36}
\end{figure}
\begin{figure}[H]
	\centering
	\includegraphics[width=16cm]{37}
\end{figure}
\begin{figure}[H]
	\centering
	\includegraphics[width=16cm]{38}
\end{figure}
\begin{figure}[H]
	\centering
	\includegraphics[width=16cm]{39}
\end{figure}
\begin{figure}[H]
	\centering
	\includegraphics[width=16cm]{40}
\end{figure}
\begin{figure}[H]
	\centering
	\includegraphics[width=16cm]{41}
\end{figure}
\begin{figure}[H]
	\centering
	\includegraphics[width=16cm]{42}
\end{figure}
\begin{figure}[H]
	\centering
	\includegraphics[width=16cm]{43}
\end{figure}
\begin{figure}[H]
	\centering
	\includegraphics[width=16cm]{44}
\end{figure}
\begin{figure}[H]
	\centering
	\includegraphics[width=16cm]{45}
\end{figure}
\begin{figure}[H]
	\centering
	\includegraphics[width=16cm]{46}
\end{figure}
\begin{figure}[H]
	\centering
	\includegraphics[width=16cm]{47}
\end{figure}
\begin{figure}[H]
	\centering
	\includegraphics[width=16cm]{48}
\end{figure}
\begin{figure}[H]
	\centering
	\includegraphics[width=16cm]{49}
\end{figure}
\begin{figure}[H]
	\centering
	\includegraphics[width=16cm]{50}
\end{figure}
\begin{figure}[H]
	\centering
	\includegraphics[width=16cm]{51}
\end{figure}
\begin{figure}[H]
	\centering
	\includegraphics[width=16cm]{52}
\end{figure}
\begin{figure}[H]
	\centering
	\includegraphics[width=16cm]{53}
\end{figure}
\begin{figure}[H]
	\centering
	\includegraphics[width=16cm]{54}
\end{figure}
\begin{figure}[H]
	\centering
	\includegraphics[width=16cm]{55}
\end{figure}
\begin{figure}[H]
	\centering
	\includegraphics[width=16cm]{56}
\end{figure}
\begin{figure}[H]
	\centering
	\includegraphics[width=16cm]{57}
\end{figure}
\begin{figure}[H]
	\centering
	\includegraphics[width=16cm]{58}
\end{figure}
\begin{figure}[H]
	\centering
	\includegraphics[width=16cm]{59}
\end{figure}
\begin{figure}[H]
	\centering
	\includegraphics[width=16cm]{60}
\end{figure}
\begin{figure}[H]
	\centering
	\includegraphics[width=16cm]{61}
\end{figure}
\begin{figure}[H]
	\centering
	\includegraphics[width=16cm]{62}
\end{figure}
\begin{figure}[H]
	\centering
	\includegraphics[width=16cm]{63}
\end{figure}
\newpage

\section{Glosario de términos}
	\begin{enumerate}
		\item \textit{Ruta de envío}: Se trata de las oficinas y almacenes por los que pasa un paquete hasta llegar al destino.
		\item \textit{Identificador de paquete}: Código que se le asigna a un paquete cuando entra al sistema de envíos de la empresa, permitiendo diferenciarlo de todos los demás y simplificar operaciones como la localización.
		\item \textit{Coste de envío}: Coste variable del servicio de envío en función del peso del paquete, dimensiones, distancia, modo de envío (certificado, urgente, etc) y condición de socio.
		\item \textit{Fase de envío}: El transporte de un paquete se divide en diversas fases: \newline - Recepción del paquete (en oficina)
		\newline - Aceptación del envío (periodo antes de que el paquete salga para su transporte para modificar el envío)
		\newline - Transporte del paquete (entre oficinas y almacenes hasta la oficina destino)
		\newline - Llegada a oficina destino
		\newline - Llegada a destino (llegada del paquete a la dirección destino)
		\item \textit{Ruta de transporte libre}: Una ruta de transporte puede estar libre o tener una furgoneta asignada.
		\item \textit{Estado de furgoneta}: Una furgoneta puede encontrarse disponible o realizando un desplazamiento, en cuyo caso incluye la ruta de transporte que está llevando a cabo.
		\item \textit{Estado de paquete}: El estado de un paquete está formado por la fase de envío en la que se encuentra, su situación y ubicación actuales y su ruta de transporte asignada.
		\item \textit{Filtro}: Herramienta externa (ej. Google Maps) que se implementa  en el sistema para comprobar si hay algún error de entrada por parte del usuario que solicita una recogida/entrega de paquete.
		\item \textit{Canal de comunicación}: Medio por el que se transmite la información sobre el envío siendo destinatario el usuario(vía e-mail o SMS).
		\item \textit{Valoraciones}: Calificación de 0 a 5 y comentarios respecto a el servicio prestado por parte de la empresa.
		\item \textit{Almacenista}: Empleado encargado de la gestión de paquetes dentro del almacén.
		\item \textit{Empleado}: Trabajador de la empresa de transportes.
		\item \textit{Oficinista}: Empleado encargado de los tramites administrativos dentro de la empresa así como la gestión general de paquetes.
		\item \textit{Almacenista}: Empleado encargado de la distribución de paquetes en los almacenes.
		\item \textit{Conductor}: Empleado encargado del transporte de paquetes entre almacenes.	
		\item \textit{Repartidor}: Empleado encargado de la recogida y entrega de paquetes entre clientes y almacenes.
		\item \textit{Sucursal}: Oficina o almacén de la empresa.
		\item \textit{Almacén}: Edificio donde se guardan y almacenan los paquetes durante su envío hasta el destino.
		\item \textit{Área personal}: Sitio web al que se accede un cliente registrado tras identificarse y con el que puede acceder a los servicios que ofrece la empresa.
 	\end{enumerate}
\newpage
%------------------------------------------------

\end{document}
